% Foundations
\newglossaryentry{lc}{name=logical connective, description={Aussagenjunktoren}}
\newglossaryentry{qua}{name=quantifier, description={Quantor}}
\newglossaryentry{un}{name=union, description={Vereinigung}}
\newglossaryentry{int}{name=intersection, description={Durchschnitt}}
\newglossaryentry{sub}{name=subset, description={Teilmenge}}
\newglossaryentry{sup}{name=superset, description={Obermenge}}
\newglossaryentry{pow}{name=powerset, description={Potenzmenge}}
\newglossaryentry{relcomp}{name=relative complement, description={Komplement, Differenz}}
\newglossaryentry{compl}{name=complement, description={Symmetrische Differenz}}
\newglossaryentry{quotset}{name=quotient set, description={Restklassenmenge, Quotienten Menge}}
\newglossaryentry{op}{name=operation, description={Verknüpfung}}
\newglossaryentry{inc}{name=increasing, description={wachsend}}
\newglossaryentry{dec}{name=decreasing, description={fallend}}
\newglossaryentry{nest}{name=nested intervals, description={Intervallschachtelung}}
\newglossaryentry{pred}{name=predicate, description={Prädikat}}
\newglossaryentry{disj}{name=disjoint, description={disjunkt}}
\newglossaryentry{preim}{name=pre-image, description={Urbild}}
\newglossaryentry{domain}{name=domain, description={Definitionsbereich}}
\newglossaryentry{codomain}{name=co-domain, description={Zielbereich}}
\newglossaryentry{part}{name=partition, description={(disjunkte) Zerlegung}}
\newglossaryentry{zerodiv}{name=no zero divisors, description={nullteiler freiheit}}

% Algebra
\newglossaryentry{eucl}{name=Euclidean Domain, description={Euklidischer Ring}}
\newglossaryentry{pre-image}{name=pre-image, description={Urbild}}
\newglossaryentry{fibre}{name=fibre, description={Faser}}
\newglossaryentry{wlog}{name=w.l.o.g., description={without loss of generality}}
\newglossaryentry{unit}{name=unit, description={Einheit}}
\newglossaryentry{coprime}{name=coprime, description={teilerfremd}}
\newglossaryentry{gcd}{name=greatest common denominator, description={Grösster gemeinsamer Teiler}}
\newglossaryentry{res}{name=residue class, description={Restklasse}}
\newglossaryentry{span}{name=(linear) span, description={Lineare Hülle}}
\newglossaryentry{transmat}{name=transformation matrix, description={Darstellende Matrix}}
\newglossaryentry{trig}{name=triangularizable, description={trigonalisierbar}}
\newglossaryentry{inv}{name=inversion, description={Fehlstand, Inversion}}
\newglossaryentry{adj}{name=adjugate matrix, description={Adjunkte, komplementäre Matrix}}
\newglossaryentry{path}{name=path-component, description={Wegzusammenhangskomponente}}
\newglossaryentry{tr}{name=trace, description={Spur}}
\newglossaryentry{comp}{name=companion matrix, description={Begleitmatrix}}
\newglossaryentry{ext}{name={Exterior Product}, description={Äussere Potenz}}
\newglossaryentry{sym}{name={Symmetric Power}, description={Symmetrische Potenz}}
\newglossaryentry{outer}{name={Outer Product}, description={Dyadisches-, Tensorielles Produkt}}
\newglossaryentry{dot}{name={Dot Product}, description={Standardskalarprodukt}}
\newglossaryentry{inner}{name={Inner Product}, description={Skalarprodukt}}
\newglossaryentry{axis}{name={Principal Axis Theorem}, description={Hauptachsentransformation}}
\newglossaryentry{adju}{name={self-adjoint}, description={selbstadjungiert}}
\newglossaryentry{sylv}{name={Sylvester's Law of Inertia}, description={Sylvesterscher Trägheitssatz}}
\newglossaryentry{cross}{name={Cross Product}, description={Kreuzprodukt, Vektorprodukt}}
\newglossaryentry{gram}{name={Gram-Schmidt Process}, description={Gram-Schmidtsches Orthogonalisierungsverfahren}}
\newglossaryentry{cos}{name={Law of Cosines}, description={Kosinussatz}}
\newglossaryentry{quot}{name={Quotient-, Factor-, Residue (Class) Ring}, description={Quotientenring, Restklassenring}}
\newglossaryentry{tors}{name={Torsion}, description={Torsion, Torsionsuntermodul}}
\newglossaryentry{cokernel}{name={Cokernel}, description={Kokern}}
\newglossaryentry{ufd}{name={Unique Factorization Domain (UFD)}, description={Faktorieller Ring}}
\newglossaryentry{pid}{name={Principal Ideal Domain (PID)}, description={Hauptidealring}}
\newglossaryentry{solv}{name={solvable group}, description={Auflösbare Gruppe}}
\newglossaryentry{inva}{name={invariant}, description={remain unchanged under changed conditions}}
