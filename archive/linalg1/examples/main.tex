\section{Groups}
A group is an algebraic structure consisting of a set of elements equipped with an operation that combines any two elements to form a third element and that satisfies four conditions called the group axioms, namely closure, associativity, identity and invertibility.

\begin{example}[Integers]
   \((\mathbb{Z}, +, 0)\) is an abelian group since:
   \begin{enumerate}
      \item \(\mathbb{Z}\) includes the 0 as the neutral element: \(\forall a \in \mathbb{Z}: a + 0 = a\)
      \item \(\mathbb{Z}\) includes negative numbers as inverse elements: \(\forall a \in \mathbb{Z}: a + (-a) = 0\)
      \item addition is associative: \(\forall a, b \in \mathbb{Z}: a + b = b + a\)
      \item addition is commutative: \(\forall a, b, c \in \mathbb{Z}: (a + b) + c = a + (b + c)\)
   \end{enumerate}

   \((\mathbb{N}_{0}, +, 0)\) however is not a group since \(\mathbb{N}_{0}\) does not include negative numbers to act as the inverse elements.
   \(\forall a, b \in \mathbb{N}_{0}: a + b = 0\)
\end{example}

\begin{example}[Rational Numbers]
   \((\mathbb{Z}, \cdot, 1)\) is not a group, since there are no fractions in \(\mathbb{Z}\) which could act as the inverse elements.

   But \((\mathbb{Q}, \cdot, 1)\) is a abelian group since \(\mathbb{Q}\) includes fractions: \(\forall a \in \mathbb{Q}: a \cdot \frac{1}{a} = 1\)

   The same group with the division \((\mathbb{Q}, /, 1)\) is not a group since the associativity is not supported: \(2 / (2 / 2) \neq (2 / 2) / 2\)
\end{example}

\begin{example}[Matrices]
   The set of matrices is a so called \textit{additive} group \((Mat_{m}(K), +, 0_m)\) since \(A-A = 0_m\) and \(A + 0_m = A\)

   \((Mat_{m}(K), \cdot, 0_m)\) is a \textit{multiplicative} group since \(A(BC)=(AB)C\) and \(AI_m = I_mA = A\) but is not abelian because matrix multiplication is not commutative \(AB \neq BA\)

   An important group of matrices is the general linear group of \cref{pro:general_linear_group}.
\end{example}

\section{Rings}
A ring is a group with an additional operation, which is compatible with the first.

\section{Fields}
A field is a ring such that the second operation also satisfies all the group properties.
This way a field consists of an \textit{additive group} and a \textit{multiplicative group} which are compatible through the distributivity.

\begin{example}[Rational Numbers]
   \((\mathbb{R}, +, \cdot, 0, 1)\) is a field since \((\mathbb{R}, +, 0)\) forms an abelian group and \((\mathbb{R}^*, \cdot, 1)\) also forms an abelian group.
   These two groups interact with each other through the distributivity of their interactions.
\end{example}

\section{Vector Spaces}
A vector space (also called a linear space) is a collection of objects called vectors, which may be added together and multiplied ("scaled") by numbers, called scalars.
Scalars are often taken to be real numbers, but there are also vector spaces with scalar multiplication by complex numbers, rational numbers, or generally any field.
Vectors in vector spaces do not necessarily have to be arrow-like objects as they appear in the mentioned examples: vectors are regarded as abstract mathematical objects with particular properties, which in some cases can be visualized as arrows.

\section{Matrices}
\subsection{Special Cases}
\begin{example}[Zero Matrix]
   \[0_{3,3} = \begin{pmatrix}
      0 & 0 & 0 \\
      0 & 0 & 0 \\
      0 & 0 & 0 \\
   \end{pmatrix}\]
\end{example}
\begin{example}[Diagonal Matrix]
   \[D_{3} = \begin{pmatrix}
         a_{11} & 0 & 0 \\
         0 & a_{22} & 0 \\
         0 & 0 & a_{33} \\
   \end{pmatrix}\]
\end{example}
\begin{example}[Identity Matrix]
   \[I_{3} = \begin{pmatrix}
         1 & 0 & 0 \\
         0 & 1 & 0 \\
         0 & 0 & 1 \\
   \end{pmatrix}\]
\end{example}
\begin{example}[Single Entry Matrix]
   \[E_{2,3} = \begin{pmatrix}
         0 & 0 & 0 \\
         0 & 0 & 1 \\
         0 & 0 & 0 \\
   \end{pmatrix}\]
\end{example}

\begin{example}[Row Echelon Form]
   For the matrix
   \[\begin{pmatrix}
         0 & 2 & 0 & 4 & 6 & 0 & 5 \\
         0 & 0 & 1 & 3 & 2 & 1 & 0 \\
         0 & 0 & 0 & 0 & 0 & 3 & 1 \\
         0 & 0 & 0 & 0 & 0 & 0 & 0 \\
   \end{pmatrix}\]
   is \(m = 4,~n = 7,~r = 3,~j_1 = 2,~j_2 = 3,~j_3 = 6\)
\end{example}

\begin{example}[Matrix Multiplication]
   \[\begin{pmatrix} 3 & 2 & 1 \\ 1 & 0 & 2 \end{pmatrix} \cdot \begin{pmatrix} 1 & 2 \\ 0 & 1 \\ 4 & 0 \end{pmatrix} = \begin{pmatrix} 7 & 8 \\ 9 & 2 \end{pmatrix}\]
\end{example}

\begin{example}[Triangular Matrix]
   \[\begin{pmatrix}
         0 & 2 & 0 & 6 \\
         0 & 0 & 1 & 2 \\
         0 & 0 & 0 & 3 \\
         0 & 0 & 0 & 0 \\
   \end{pmatrix}~\begin{pmatrix}
         0 & 0 & 0 & 0 \\
         0 & 0 & 0 & 0 \\
         2 & 1 & 0 & 0 \\
         5 & 2 & 3 & 0 \\
   \end{pmatrix}\]
\end{example}

\begin{example}[Permutationmatrix]
   \[\begin{pmatrix}
         0 & 0 & 0 & 1 & 0 \\
         0 & 1 & 0 & 0 & 0 \\
         1 & 0 & 0 & 0 & 0 \\
         0 & 0 & 0 & 0 & 1 \\
         0 & 0 & 1 & 0 & 0
   \end{pmatrix}\]
\end{example}

\begin{example}[Transposed Matrix]
   \[\begin{pmatrix} 1 & 4 \\ 8 & -2 \\ -3 & 5\end{pmatrix}^T = \begin{pmatrix} 1 & 8 & -3 \\ 4 & -2 & 5 \end{pmatrix}\]
\end{example}

\section{Linear Equations}
\section{Linear Maps}
A linear map describes the relationship between distinct sets. Given two sets X and Y a linear map f is a rule that maps every \(x \in X\) uniquely to a \(f(x) \in Y\).
\section{Mapping Spaces}

\subsection{Transformation Matrix}
We need ordere bases for transformation matrices, since the order of base-vectors defines how the calculation of the transformation matrix is carried out. a set as a bases wouldnt give us this.

\section{Relations}
Jede Äquivalenzrelation R auf einer Menge X liefert also eine zerlegung von X in disjunkte Äquivalenzklassen.
Diese Äquivalenzklassen betrachtet man als Elemente einer neuen Menge, die mit X/R bezeichnet wird.
Man nennt sie die Quotientenmenge von X anch der Äquivalenzrelation R

\section{Basis}
die schnittmenge einer linear unabhängige teilmenge eines vektorraums und eines erzeugendensystems desselben vektorraums ist die basis.

\subsection{span}
Das Erzeugnis ist die Menge aller vektoren eines Vektorraums, die sich aus einer endlichen Teilfamilie von \((v_i)_{i \in I}\) linear kombinieren lassen.
die teilfamilie entsteht dadurch, dass fast alle koeffizienten der linear kombination 0 sind, ausser die derjenigen vektoren welche für die linearkombination benötigtwerden.
