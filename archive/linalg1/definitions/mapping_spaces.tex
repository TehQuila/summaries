\section{Mapping Spaces}
\begin{definition}[Homomorphism Space]
   Given K-vector spaces \(V, W\) with \(\text{dim}V = n,~\text{dim}W = m\)
   \[\text{Hom}_K(V, W) := \{f: V \to W \mid f~\text{is linear}\}\]
   is the space of all linear mappings \(V \to W\)
\end{definition}

% todo: actually lemma?
\begin{lemma}
   Given the set \(W^V\) of all mappings \(V \to W\)
   \[\text{Hom}_K(V, W) \subset W^V\]
   is a vector subspace.
\end{lemma}
% todo: missing proof
\begin{proof}
\end{proof}
% todo: what is this?
for \(f: V' \to W,~g: W \to W'\) is \(C_{g \circ f}: \text{Hom}_K(V, W) \to \text{Hom}_K(V', W'),~h \mapsto g \circ h \circ f\)

\begin{lemma}
   Given the vector space \(\text{Mat}_{m,n}(K)\) and \(\text{Hom}_K(K^n, K^m)\) there is the isomorphism
   \[\text{Mat}_{m,n}(K) \xrightarrow{\sim} \text{Hom}_K(K^n, K^m)\]
   \[A_{m,n} \mapsto L_A\]
\end{lemma}
\begin{proof}
   The Mapping is an isomorphism because it is bijective and between vector spaces. It is bijective since it is

   injective according to \cref{pro:kerf=0}

   surjective according to \cref{def:surjective}
\end{proof}

\begin{lemma}
   Given the ordered bases \(B = (v_1, \ldots, v_n)\) of \(V\) and \(B' = (w_1, \ldots, w_n)\) of \(W\)
   \[\text{Hom}_K(V, W) \xrightarrow{\sim} \text{Mat}_{m,n}(K)\]
   \[f \mapsto {}_{B'}M_{B}(f)\]
\end{lemma}
\begin{proof}
   There exists an inverse mapping \(A \mapsto h\) where \(h(v_j) = \sum a_{ij} w_i\) which means it is bijective.
\end{proof}
% todo: actually prop?
\begin{proposition}[\(\text{dim}(\text{Hom}_K(V, W)) = m \cdot n\)]\label{pro:hom_dim}
   from those two lemmas follows that \(\text{dim}(\text{Hom}_K(V, W)) = m \cdot n\) (see def)
\end{proposition}
\begin{remark}
   Dimensions of homomorphism are the same (since isomorphisms are bijective?)
   \(\text{Hom}_K(V, W) \cong \text{Hom}_K(K^n, K^m) \cong \text{Mat}_{m,n}(K)\)
\end{remark}

\begin{definition}[Endomorphism Space]
   The homomorphism space of a K-vector space \(V\) and itself
   \[\text{End}_K(V) := \text{Hom}_K(V, V)\]
\end{definition}
\begin{remark}
   \(\text{Hom}_K(V, V) \cong \text{Mat}_n(K)\)
   \[(\text{End}_K(V), \circ, +, 0_V, \text{id}_V)\]
   is the endomorphism ring of \(V\)
\end{remark}

\begin{definition}[Dual Space]
   For a K-vector space \(V\)
   \[V^* := \text{Hom}_K(V, K)\]
   is its dual space.

   \(l \in V^*\) is a \textit{linear form}
\end{definition}
\begin{remark}
   \(\text{dim}V^* = \text{dim}V\)
\end{remark}

% todo: dual basis
\begin{proposition}[Dual Basis]
   Given an ordered basis \(B\) or a K-vector space \(V\) and \(l_i \in V^*~\forall 1 \leq i \leq n\) with \(l_i: V \to K,~\sum_{j=1}^n \alpha_j v_j \mapsto \alpha_i \iff l_i(v_i) = \delta_{íj}\)
   \[B^* = (l_1, \ldots, l_n)\]
   is the ordered Basis of \(V^*\), called the dual basis to \(B\)
\end{proposition}
\begin{proof}
   It is sufficient to show that \((l_1, \ldots, l_n)\) are linear independent in order to show that \(B^*\) is a basis.
   According to \cref{pro:hom_dim} is
   \[\text{dim}V^* = n \cdot 1\]

   Suppose \(\exists j: y_j \neq 0\)
   \[\sum_{i=1}^n y_i l_i(v_j) = y_j \implies y_j \neq 0\]
   which is a contradiction
\end{proof}

\begin{proposition}
   Given ordered bases \(B, B'\) of two K-vector spaces \(V, W\)
   \(\forall f: V \to W\)
   \[\exists f^*: W^* \to V^*\]
   \[l \mapsto l \circ f\]
   is a dual map to \(f\)

   \[{}_{B}M_{B'}(f^*) = {}_{B'}M_{B}^T(f)\]
   characterises the dual map entirely
\end{proposition}
\begin{proof}[Proof (Matrix of \(l \circ f = \text{Mat}^T\))]
   Given \(l \circ f \in V^*: (V \xrightarrow{f} W \xrightarrow{l} K)\)

   \[{}_{B}M_{B'}(f) = (a_{ij})_{\substack{1 \leq i \leq m \\ 1 \leq j \leq n}}\]
   \[f(v_j) = \sum_{i=1}^m a_{ij} w_i\]

   \[B^* = (l_1, \ldots, l_n)~\text{of}~V^*\]
   \[B^*' = (l_1', \ldots, l_n')~\text{of}~W^*\]

   \[l_K' \circ f = \sum_{i=1}^n M_{ik} l_i\]
   \[l_K' \circ f(v_j) = l_K'\left(\sum_{i=1}^m a_{ij} w_i \right) = a_{kj}\]
   \[l_K' \circ f = \sum_{i=1}^n a_{ki} l_i = \sum_{i=1}^m (a^T)_{ik} l_i \implies A^T = {}_{B^*}M_{B^*'}(f^*)\]
\end{proof}

\begin{proposition}
   \(\forall v \in V\) exists an Evaluation
   \[w_V: V^* \to K\]A
   \[l \mapsto l(v)\]

   \[w_V \in (V^*)^*~\text{with}~\text{dim}V^{**} = \text{dim}V\]
\end{proposition}

\begin{definition}[Automorphism Set]
   Given the K-vector space \(V\)
   \[\text{Aut}_K(V) := \{f: V \to V \mid f~\text{is isomorphic}\}\]
   is the set of all automorphisms of \(V\).
\begin{lemma}
   \(\text{Aut}_K(V)\) is a group.
\end{lemma}
\begin{proof}
   \[\circ = f\]
   \[e = \text{id}\]
   \[\text{There exists an inverse element since}~f~\text{is an isomorphism.}\]
\end{proof}

