\section{Rings}
\begin{definition}[Ring]
   A set \(R\) with two operations
   \[+: R \times R \to R,~(a, b) \mapsto a + b\]
   \[\cdot: R \times R \to R,~(a, b) \mapsto a \cdot b\]
   if the following holds:
   \begin{enumerate}[label=(\roman*)]
      \item \((R, +, 0)\) is an abelian group, with \(a^{-1} := -a\).
      \item Multiplication is associative: \(\forall a, b, c \in R: (a \cdot b) \cdot c = a \cdot (b \cdot c)\)
      \item The operations are distributive: \(\forall a, b, c \in R: a \cdot (b + c) = ab + bc \land (a + b) \cdot c = ac + bc\)
   \end{enumerate}
   Can be written as \((R, +, \cdot, 0, 1)\).
\end{definition}

% todo: write down rings
\subsection{Unitary Ring}
If the ring includes a neutral element of the multiplication it is a unitary ring (eins-ring, unitär ring).
Ex: \((\mathbb{Z}, +, \cdot, 0, 1)\)

\subsection{Examples}
\((\mathbb{Z}, +, \cdot, 0, 1)\) is a ring since \((\mathbb{Z}, +, 0)\) is an abelian group and distributivity is given.
\((Mat_{m}(K), +, \cdot, 0_m, I_m)\) is a unitary ring but no field (no inverse)
