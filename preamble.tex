\documentclass[english,titlepage]{uzhpub}
% Font Encoding
\usepackage{lmodern}
\usepackage[T1]{fontenc}
\usepackage[utf8]{inputenc}
\input{glyphtounicode}
\pdfgentounicode=1

% Glossaries
\usepackage[toc,nonumberlist]{glossaries}
% Foundations
\newglossaryentry{lc}{name=logical connective, description={Aussagenjunktoren}}
\newglossaryentry{qua}{name=quantifier, description={Quantor}}
\newglossaryentry{un}{name=union, description={Vereinigung}}
\newglossaryentry{int}{name=intersection, description={Durchschnitt}}
\newglossaryentry{sub}{name=subset, description={Teilmenge}}
\newglossaryentry{sup}{name=superset, description={Obermenge}}
\newglossaryentry{pow}{name=powerset, description={Potenzmenge}}
\newglossaryentry{relcomp}{name=relative complement, description={Komplement, Differenz}}
\newglossaryentry{compl}{name=complement, description={Symmetrische Differenz}}
\newglossaryentry{quotset}{name=quotient set, description={Restklassenmenge, Quotienten Menge}}
\newglossaryentry{op}{name=operation, description={Verknüpfung}}
\newglossaryentry{inc}{name=increasing, description={wachsend}}
\newglossaryentry{dec}{name=decreasing, description={fallend}}
\newglossaryentry{nest}{name=nested intervals, description={Intervallschachtelung}}
\newglossaryentry{pred}{name=predicate, description={Prädikat}}
\newglossaryentry{disj}{name=disjoint, description={disjunkt}}
\newglossaryentry{preim}{name=pre-image, description={Urbild}}
\newglossaryentry{domain}{name=domain, description={Definitionsbereich}}
\newglossaryentry{codomain}{name=co-domain, description={Zielbereich}}
\newglossaryentry{part}{name=partition, description={(disjunkte) Zerlegung}}
\newglossaryentry{zerodiv}{name=no zero divisors, description={nullteiler freiheit}}

% Algebra
\newglossaryentry{eucl}{name=Euclidean Domain, description={Euklidischer Ring}}
\newglossaryentry{pre-image}{name=pre-image, description={Urbild}}
\newglossaryentry{fibre}{name=fibre, description={Faser}}
\newglossaryentry{wlog}{name=w.l.o.g., description={without loss of generality}}
\newglossaryentry{unit}{name=unit, description={Einheit}}
\newglossaryentry{coprime}{name=coprime, description={teilerfremd}}
\newglossaryentry{gcd}{name=greatest common denominator, description={Grösster gemeinsamer Teiler}}
\newglossaryentry{res}{name=residue class, description={Restklasse}}
\newglossaryentry{span}{name=(linear) span, description={Lineare Hülle}}
\newglossaryentry{transmat}{name=transformation matrix, description={Darstellende Matrix}}
\newglossaryentry{trig}{name=triangularizable, description={trigonalisierbar}}
\newglossaryentry{inv}{name=inversion, description={Fehlstand, Inversion}}
\newglossaryentry{adj}{name=adjugate matrix, description={Adjunkte, komplementäre Matrix}}
\newglossaryentry{path}{name=path-component, description={Wegzusammenhangskomponente}}
\newglossaryentry{tr}{name=trace, description={Spur}}
\newglossaryentry{comp}{name=companion matrix, description={Begleitmatrix}}
\newglossaryentry{ext}{name={Exterior Product}, description={Äussere Potenz}}
\newglossaryentry{sym}{name={Symmetric Power}, description={Symmetrische Potenz}}
\newglossaryentry{outer}{name={Outer Product}, description={Dyadisches-, Tensorielles Produkt}}
\newglossaryentry{dot}{name={Dot Product}, description={Standardskalarprodukt}}
\newglossaryentry{inner}{name={Inner Product}, description={Skalarprodukt}}
\newglossaryentry{axis}{name={Principal Axis Theorem}, description={Hauptachsentransformation}}
\newglossaryentry{adju}{name={self-adjoint}, description={selbstadjungiert}}
\newglossaryentry{sylv}{name={Sylvester's Law of Inertia}, description={Sylvesterscher Trägheitssatz}}
\newglossaryentry{cross}{name={Cross Product}, description={Kreuzprodukt, Vektorprodukt}}
\newglossaryentry{gram}{name={Gram-Schmidt Process}, description={Gram-Schmidtsches Orthogonalisierungsverfahren}}
\newglossaryentry{cos}{name={Law of Cosines}, description={Kosinussatz}}
\newglossaryentry{quot}{name={Quotient-, Factor-, Residue (Class) Ring}, description={Quotientenring, Restklassenring}}
\newglossaryentry{tors}{name={Torsion}, description={Torsion, Torsionsuntermodul}}
\newglossaryentry{cokernel}{name={Cokernel}, description={Kokern}}
\newglossaryentry{ufd}{name={Unique Factorization Domain (UFD)}, description={Faktorieller Ring}}
\newglossaryentry{pid}{name={Principal Ideal Domain (PID)}, description={Hauptidealring}}
\newglossaryentry{solv}{name={solvable group}, description={Auflösbare Gruppe}}
\newglossaryentry{inva}{name={invariant}, description={remain unchanged under changed conditions}}

\makeglossaries
\glsaddall

% enumerate Environment
\usepackage{enumitem}
\SetLabelAlign{Center}{\hfil(#1)\hfil} % center alignment for enumerate

% Math Packages
\usepackage{mathtools} % includes amsmath
\usepackage{amsthm} % for theoremstyle, newtheorem
\usepackage{amsfonts} % for mathbb, mathcal
\usepackage{amssymb} % for blacksquare
\usepackage{thmtools} % for theorem list
\usepackage{mathrsfs} % for mathscr font

% Drawing Packages
\usepackage{pgf}
\usepackage{pgfplots}
\pgfplotsset{compat=1.15}
\usepackage{tikz}
\usetikzlibrary{arrows}
\usepackage{tikz-cd}
\usepackage[customcolors]{hf-tikz} % equation styling

% Code Snippets
\usepackage{listings}
\lstset{inputpath=codes}
\lstset{
   identifierstyle=\color{red},
   keywordstyle=\color{blue},     % keyword style
   commentstyle=\color{grey},     % comment style
   stringstyle=\color{green},     % string literal style
   numberstyle=\tiny,             % the style that is used for the line-numbers
   basicstyle=\footnotesize,      % the size of the fonts that are used for the code
   breakatwhitespace=false,       % sets if automatic breaks should only happen at whitespace
   breaklines=true,               % sets automatic line breaking
   captionpos=b,                  % sets the caption-position to bottom
   keepspaces=true,               % keeps spaces in text, useful for keeping indentation of code (possibly needs columns=flexible)
   numbers=left,                  % where to put the line-numbers; possible values are (none, left, right)
   stepnumber=1,                  % the step between two line-numbers. If it's 1, each line will be numbered
   rulecolor=\color{black},       % if not set, the frame-color may be changed on line-breaks within not-black text (e.g. comments (green here))
   showspaces=false,              % show spaces everywhere adding particular underscores; it overrides 'showstringspaces'
   showstringspaces=false,        % underline spaces within strings only
   showtabs=false,                % show tabs within strings adding particular underscores
   tabsize=1,                     % sets default tabsize to 2 spaces
   title=\lstname,
   xleftmargin=7ex
}

% base16 color scheme
\usepackage{xcolor}
\definecolor{black}{HTML}{383838}
\definecolor{grey}{HTML}{787878}
\definecolor{white}{HTML}{D8D8D8}
\definecolor{red}{HTML}{AB4642}
\definecolor{orange}{HTML}{DC9656}
\definecolor{yellow}{HTML}{F7CA88}
\definecolor{green}{HTML}{A1B56C}
\definecolor{cyan}{HTML}{86C1B9}
\definecolor{blue}{HTML}{7CAFC2}
\definecolor{purple}{HTML}{BA8BAF}
\definecolor{brown}{HTML}{A16946}

% Referencing
\usepackage{xr} % cross file referencing
\usepackage{hyperref} % make references to hyperlinks
% reset section counter for parts
\makeatletter
\@addtoreset{section}{part}
\makeatother
\usepackage{cleveref} % prepend environment before reference number
% LABELLING SCHEME
% ch: chapter
% sec: section
% ssec: subsection
% fig: figure
% tab: table
% eq: equation
% lst: code listing
% itm: enumerated list item
% alg: algorithm
% app: appendix subsection

% def: definition
% thm: theorem
% pro: proposition
% cor: corollary
% lem: lemma
% rem: remark
% ex: example

% Math styling
\renewcommand\qedsymbol{$\blacksquare$}
\usepackage{mdframed}
\newmdtheoremenv[linewidth=1, linecolor=black, backgroundcolor=white]{definition}{Definition}[section]
\newmdtheoremenv[linewidth=1, linecolor=black, backgroundcolor=white]{proposition}[definition]{Proposition}
\newmdtheoremenv[linewidth=1, linecolor=black, backgroundcolor=white]{lemma}[definition]{Lemma}
\newmdtheoremenv[linewidth=1, linecolor=black, backgroundcolor=white]{theorem}[definition]{Theorem}
\newmdtheoremenv[linewidth=1, linecolor=black, backgroundcolor=white]{corollary}{Corollary}[theorem]
\theoremstyle{remark}
\newtheorem*{remark}{Remark}
\theoremstyle{example}
\newtheorem*{example}{Example}

\numberwithin{equation}{section}

% Custom Math Commands

% Foundations
\DeclareMathOperator\sign{sign}
\DeclareMathOperator\im{im}
\DeclareMathOperator\id{id}
\DeclareMathOperator\Abb{Abb}
\DeclareMathOperator\Ran{Ran}

% Algebra
\DeclareMathOperator\adj{adj}
\DeclareMathOperator\tr{tr}
\DeclareMathOperator\inv{inv}
\DeclareMathOperator\spanv{span}
\DeclareMathOperator\Aut{Aut}
\DeclareMathOperator\Hom{Hom}
\DeclareMathOperator\End{End}
\DeclareMathOperator\Eig{Eig}
\DeclareMathOperator\Mat{Mat}
\DeclareMathOperator\SR{SR}
\DeclareMathOperator\ZR{ZR}
\DeclareMathOperator\rk{rk}
\DeclareMathOperator\GL{GL}
\DeclareMathOperator\Sym{Sym}
\DeclareMathOperator\chark{char}
\DeclareMathOperator\OG{O}
\DeclareMathOperator\SO{SO}
\DeclareMathOperator\U{U}
\DeclareMathOperator\SU{SU}
\DeclareMathOperator\proj{proj}
\DeclareMathOperator\coker{coker}
\DeclareMathOperator\tors{tors}
\DeclareMathOperator\pr{pr}
\DeclareMathOperator\ev{ev}
\DeclareMathOperator\Perm{Perm}
\DeclareMathOperator\trdeg{trdeg}

% Analysis
\DeclareMathOperator\arctanh{arctanh}
\DeclareMathOperator\arcsinh{arcsinh}
\DeclareMathOperator\arccosh{arccosh}
\DeclareMathOperator\ixp{ixp}
\DeclareMathOperator\argu{arg}

% Numerics
\newcommand{\imag}{\mathrm{i}}
\DeclareMathOperator\suppf{suppf}
\DeclareMathOperator\supp{supp}
\DeclareMathOperator\opt{opt}
\DeclareMathOperator\chop{chop}
\DeclareMathOperator\fl{fl}
\DeclareMathOperator\rd{rd}
\DeclareMathOperator\eps{eps}

% Custom Math Commands
\providecommand{\abs}[1]{\left\lvert#1\right\rvert}
\providecommand{\norm}[1]{\left\lVert#1\right\rVert}
